% Created 2018-08-11 Sat 15:49
% Intended LaTeX compiler: pdflatex
\documentclass[9pt]{article}
\usepackage[utf8]{inputenc}
\usepackage[T1]{fontenc}
\usepackage{graphicx}
\usepackage{grffile}
\usepackage{longtable}
\usepackage{wrapfig}
\usepackage{rotating}
\usepackage[normalem]{ulem}
\usepackage{amsmath}
\usepackage{textcomp}
\usepackage{amssymb}
\usepackage{capt-of}
\usepackage{hyperref}
\usepackage{tabularx}
\usepackage[table]{xcolor}
\usepackage[margin=2cm]{geometry}
\author{Boyang}
\date{\today}
\title{Cross-language Sentiment Analysis for Social Media}
\hypersetup{
 pdfauthor={Boyang},
 pdftitle={Testing in Cross-language Sentiment Analysis},
 pdfkeywords={},
 pdfsubject={},
 pdfcreator={Emacs 26.1 (Org mode 9.1.13)}, 
 pdflang={English}}
\begin{document}

\maketitle

\section{Backgrund}
Social Media becomes more and more popular. There are lots of text comments for discussion different topics everday.
However, all of those topics have discusse by different language spearkers, which have different culture backgraound.
Most of people have language and culture barrier when we doing cross-language communicate with each other.
Manpower is impossible to analysis huge amount of data. So,
We want to using machine trainslation and sentiment analysis tools to solve this problems.
Let people can understand different language speakers' attitudes (positive or negative), emotions and opinions.\\
\underline{\textbf{(Please, finding some of paper to support the backgraound)}}
\section{Purpose}
The purpose of this research is to achieve a method for finding which machine translator service combined with which English sentiment analysis service can obtain reliable sentiment analysis result for non-English speaker, who does not have sentiment analysis tool to analysis their own language.

\section{Todo List}
\subsection{Timetable}
\begin{center}

\begin{tabularx}{\textwidth}{X|X}
  \textbf{Deadline} & \textbf{Part} \\
\hline
20/08/2018 & Annotated bibliography and literature review\\
23/08/2018 & Research Proposal\\
27/08/2018 & Survey - Questionnaires\\
03/09/2018 & Qualitative analysis\\
\end{tabularx}

\end{center}
\subsubsection{detecting advertising in BBS comments (Chinese and English)}
\begin{itemize}
\item fouce on detecting Chinese advertising (Xudong and Xiaoxia can choose this part)
\end{itemize}
detecting advertising in BBS comments or Movies Review, etc
Thinking and research How to measure the quality of detecting advertising modle

\begin{itemize}
\item detecting English advertising
\end{itemize}

\subsubsection{sentiment analysis}
\begin{itemize}
\item what is sentiment analysis: Sentiment analysis model for recognize the test data belong to positive, negative or neutral classification
\item fouce on creating English sentiment analysis model first (using Natural Language Toolkit (NLTK))
\item Others Libary for using on natural language processing: Apache OpenNLP, Stanford NLP suite, Gate NLP library or Gensim.
\item when you finish creating English sentiment analysis model, Thinking about to measure the quality of your model. Testing each part of your model.
\end{itemize}

\begin{enumerate}
\item step of sentiment analysis (English)
\begin{enumerate}
\item Texting segmentation for the separate sentence to a list of keywords
\item Filtering for removing unnecessary keywords that will not add value to sentiment analysis, Such as is, but, it and so on.
\item Finding the basic words for Convert all infections to their root words.
\item Making Features for Using the root words as features to indicate the positiveness or negativeness
\item The last step is Classifier for the train a classifier to predict positivity.
\end{enumerate}
\rowcolors{1}{green}{pink}
\begin{tabularx}{\textwidth}{p{1cm}|X|X|p{2cm}|p{2cm}}
  \textbf{Part} & \textbf{Description} & \textbf{Comment} & \textbf{Undertaker} & \textbf{Signature} \\
  \hline
  Part 1 & Backgrund and detecting advertising & show current social media usage
  amount, demonstrate most of researcher fouce on English sentiment analysis, no
  much people doing non-English sentiment analysis. Detecting Chinese
  advertising & Xiaoxia Pu & \\ \hline
Part 2 & sentiment analysis step 1, 2 and 3 &  & Helene Tran & \\ \hline
Part 3 & sentiment analysis step 4 and 5 &  & Xudong Zhang & \\ \hline
Part 4 & Mechine Translation and Sentiment Analysis Testing Modle &  & Boyang Yan & \\ \hline
\end{tabularx}
\end{enumerate}
\end{document}